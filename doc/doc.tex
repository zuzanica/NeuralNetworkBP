\documentclass[a4paper, 11pt]{article}
\usepackage[left=2cm,text={17cm, 24cm},top=3cm]{geometry}
\usepackage[utf8x]{inputenc} 
\usepackage[czech]{babel}
\usepackage[IL2]{fontenc}
\providecommand{\uv}[1]{\quotedblbase #1\textquotedblleft}
\usepackage{algorithmic}
\usepackage{graphics}
\usepackage{picture}
\usepackage[ruled,longend,noline,czech,linesnumbered,]{algorithm2e}
\usepackage{multirow}
\usepackage{hyperref}

\begin{document}

\begin{titlepage}
\begin{center}
\textsc{\Huge{Vysoké učení technické v~Brně}\\
\medskip
\huge {Fakulta informačních technologií}}\\
\vspace{\stretch{0.382}}
\LARGE{Soft Computing\,--\ projekt}\\
\Huge{Neurónová sieť Backpropagation }
\vspace{\stretch{0.618}}
\end{center}
{\Large \today \hfill Zuzana Studená}
\end{titlepage}

\newpage

\section{Úvod}
Projekt sa zaoberá návrhom a implemenátáciou neurónovej siete využitím algoritmu Backpropagation. Prvá časť popisuje stručne problematiku algoritmu Backpropagation. Druhá časť popisuje vytvorený program, ktorý ukazuje funkčnosť neurónovej siete. V sekcii \ref{testavys} sú zhrnuté dosiahnuté výsledky pri trénovaní a testovaní na dátovej sade pšeničných jadier, ktoré sú priraďované do jednotlivých kategórií podla rôznych parametrov.

\section{Backpropagation}

\section{Implementácia}
Program je implementovaný v jazyku java. Implementácia sa zaoberá spracovaním configuračného resp. dátového súboru a samotnou implmentáciou neurónovej siete Backpropagation.  
\subsection{Spracovanie vstupu a inicializácia}
Po spustený program spracováva súbor \texttt{config.properties}, v tomto súbore sú uložené všetky inicializačné údaje neurónovej siete. V prípade, keď súbor neexistuje sieť sa nastaví na predvolené údaje v zdrojovom kóde. Program spracuje trénovací a testovacia dátová sada podla uvedeného súboru \texttt{csv}. Ako poslednú hodnotu každého riadku v súbore \texttt{csv}, očakáva výsledok trénovania. Pred spustením program prevedie normalizáciu výsledkov. Každý výsledok je prevedený do intervalu $<0,1>$ kvôli aktivačnej funkcii, ktorá vracia výsledky v tomto intervale. 

\subsection{Implementácia algoritmu}
Algoritmus je implementovaný v triede \texttt{Network.java}. Tento súbor reprezentuje celú sieť tvorená neurónmi a synaprickými vláknamy. Neurón reprezentuje trieda \texttt{Neuron.java}, v ktorej má každý neurón pole synaptických váh smerujúcich do neurónu. Neurón si tiež udržuje informáciu o svojom vstupe, výstupe a aktuálnej derivácií, ktorá je potrebná pre implementáciu Backpropagation algoritmu.
Celá sieť sa skladaná zo vstupu, na ktorom sú vstupné hodnoty zo spracovaného súboru \texttt{csv}, skrytých vrstiev a výstupnej vrstvy. V každej vrstve môže byť rôzny počet neurónov. V testovacom prípade budem uvažovať iba jednu skrytú vrstvu a jeden výstupný neurón.\\
Po spustený programu a inicializácii siete prebieha učenie pre celú trénovaciu množinu. Prvky trénovacej množiny sú vyberané náhodne. Pri implimentácii sa ukázalo, že sekvenčným vyberaním prvkov z trénovacej množiny dochádzalo veľkým nepresnostiam učenia. Náhodný výber prvkom trénovacej množiny došlo k výraznému zlepšeniu výsledku. Sieť opakuje učenie dokým neprebehne maximálny počet epoch alebo učenie siete nedosiahne požadovanej presnosti.\\
Učenie pozostáva z troch krokov:
\begin{itemize}
\item forward propagate
\item derivation computing
\item back propagation
\end{itemize}
\textbf{Forward propagate} reprezentuje počítanie celkového výstupu siete. Pre každú vrstvu a každý neurón vrstvy je vypočítaný vstup a výstup podla algoritmu \ref{forvard}.\\
\textbf{Derivation computing} na základe vypočítaných hodnôt v predošlom kroku počíta pre každý neurón deriváciu.\\
\textbf{Back propagation} využíva vypočítaných derivácii k výpočtu zmeny synaptických váh neurónov. V tejto časti sú upravené všetky váhy neurónovej siete.\\  
Testovanie siete spočíva v odtestovaní celej testovacej množiny. Po každom teste sieť vyhodnotí testovaný vstup na základe cieľu, ktorý je zadaný spolu so vstupnými dátami v \texttt{csv} súbore. Úspešnosť siete je vypočítaný ako podiel správne vyhodnotených vstupov a počtu prvkov trénovacej množiny. Testovacia časť vykonáva iba krok \textbf{forward propagate}.  
%Pre zistenie predikcie na základe stanovených výsledkov sa používa testovacia množina. Táto množina rovnako ako trénovacia 

\section{Testovanie a výsledky}
\label{testavys}
Testovanie prebehlo na dátovej sade dostupnom z\footnotetext[1]{http://archive.ics.uci.edu/ml/datasets/seeds}. Dátová sada merá geometrické vlastnosti jadier pšenice, ktoré patria do rôznych odrôd pšenice (Kama, Rosa, a Canadian). Pre každý druh pšenice bolo náhodne vybratých sedemdesiat prvkov. Na základe röntgenovej techniky bolo možné stanovať 7 posudzovacích kritérií: \\
\begin{enumerate}
\item plocha $A$, 
\item obvod $P$, 
\item kompaktnosť $C = 4*\pi*A/P^2$, 
\item dĺžka jadra,
\item šírka jadra, 
\item koeficient asymetrie
\item dĺžka drážky jadra. 
\end{enumerate}

Pre výsledky bolo s vyššie uvedenej trénovacej množiny odobratých tridsať náhodných prvkov, desať z každého druhu pšenice. Tieto prvky boli použité ako testovacia množina neurónovej siete. Sieť má predpovedať odrodu pšenice na základe uvedených vlastností. Odroda pšenice v testovacej dátovej sade je použitá pre spočítanie úspešnosti výpočtu neurónovej siete.   \\
Neurónová sieť bola nastavená nasledovne:\\
\\
počet skrytých vrstiev : 1\\
počet skrytých neurónov : 15\\
počet výstupných neurónov  : 1\\
počet cyklov  : 300000\\
koeficient učenia=0.7\\
trénovací dataset : seeds\_trained.csv \\
testovací dataset : seeds\_tested.csv \\

Výsleledok testu je v súbore \texttt{seedsout.txt}. Test ukázal úspešnosť predikcie správnej odrody pšenice s úspešnosťou 83.33%.  


\section{Manuál}
Obsah \texttt{zip} súboru:\\
\texttt{src} \--\ zdrojové súbory programu\\
\texttt{bin} \--\ preložené súbory\\
\texttt{conf} \--\ konfiguračné a dátové súbory\\
\texttt{build.xml} \--\ prekladací súbor\\
\texttt{bpnetwork.jar} \--\ spustiteľný jar archív\\

Program je vytvorená ako aplikácie pre príkazový riadok alebo terminál. V adresári \texttt{conf} sa nachádza súbor \texttt{config.properies}. Tento súbor slúži ako konfiguračný súbor programu.\\ Obsah súboru: \\
\\
\texttt{InputsCount=7\\
HiddenLayers=1\\
HiddenNeurons=15\\
OutputNeutron=1\\
Epochs=300000\\
EPS=0.001\\
LearningRate=0.7\\
TrainingFile=conf/seeds\_trained.csv \\
TestingFile=conf/seeds\_test\_shor.csv \\
OutputFile=out.txt}\\
\\
Úpravou súboru je možné meniť nastavenie programu napr.: \\
\texttt{InputsCount=2\\
HiddenNeurons=5\\
Epochs=3000\\
LearningRate=0.5\\
TrainingFile=conf/xor.csv \\
TestingFile=conf/xor.csv}\\

\subsection*{Preklad}
Program nie je nutné prekladať v archíve je priložený spustiteľný súbor \texttt{bpnetwork.jar}. Prípadný preklad je možné pomocou \texttt{ant-u}:\\
\texttt{ant compile}


\subsection*{Spustenie}
\texttt{ant run}\\
prípadne:\\
\texttt{java -jar bpnetwork.jar}\\
\\
Výstup programu je zobrazený do príkazového riadku a zároveň do súboru špecifikovaného v \texttt{config.properies}.


\section{Záver}
 




	
\end{document}