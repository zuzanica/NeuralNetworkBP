\documentclass[a4paper, 11pt]{article}
\usepackage[left=2cm,text={17cm, 24cm},top=2.5cm]{geometry}
\usepackage[utf8x]{inputenc} 
%\usepackage[czech]{babel}
\usepackage{graphicx}
%\usepackage[IL2]{fontenc}
\providecommand{\uv}[1]{\quotedblbase #1\textquotedblleft}
%\usepackage{times}
\usepackage{hyperref}
\begin{document}

\section{Backpropagation Networks}
%TODO
\subsection{Forward Pass}
%TODO
\subsubsection*{Net Input}
%TODO
\subsubsection*{Node Output}
%TODO
\subsection{Backward Pass}
%TODO
\subsubsection*{Delta výpočet}
%TODO
	-hidden\\
	-output
\subsubsection*{Zmena váh}
%TODO
\section{Manuál k programu}
%TODO
\subsection{Abstrakcia a modely}
%TODO
\subsection{Konfigurácia a spustenie}
Program možno preložiť a spustiť pomocou pravidiel v subore Makefile, manuálnym definovaním argumentov alebo pomocou konfiguračného suboru.
\subsubsection*{Makefile}
Pre zjednodušenie manipulácie s programom je tento program možno jednoducho preložiť použitím príkazu:\\
 \texttt{make}\\
 Pre spustenie programu s prednastavenými hodnotami je možné použiť príkaz:\\
 \texttt{make run}
\subsubsection*{Argumenty programu}
\begin{itemize}
	\item[] Hidden Layer Number - celkový počet skrytých vrstiev nerátajuc vstupnu a výstupnu vrstvu
	\item[] Nodes Per Layer - celkový počet neuronov v jednej vrstve
	\item[] Learning rate - konštanta učenia používaná pri zmene váh
	\item[] Epochs to train - počet cyklov behu siete nad datasetom
	\item[] Verbosity - granularita výstupu
	\item[] Number of Inputs - počet vstupov na základe datasetu
	\item[] Datafile Path - cesta ku datasetu
%TODO Testfile Path
\end{itemize}
\subsubsection*{Konfiguračný súbor}
Konfiguračný subor \textit{configuration.cfg}, prípadne iný manuálne vytvorený špecifikuje všetky argumenty programu v tvare argument=hodnota. Je potrebné nevkladať pred hodnotu medzery a dodržať poradie argumentov. Spustenie programu s konfiguračným suborom je možné pomocou príkazu:\\
\texttt{./bpnetwork configuration.cfg}
\subsection{Implementačné detaily}
Program je implementovaný v jazyku C\texttt{++} s využitím štandartu C\texttt{++}11. Používa .csv subor pre konfiguráciu a ako vstupný dataset.
\subsubsection*{Symmetry breaking}
Pri inicializácií váh siete dochádza k javu kedy sa sieť nemôže učiť pretože zmena pre všetky váhy je rovnaká. Tento jav ma za výsledok konštantný výstup, často okolo strednej hodnoty. Riešením je práve symmetry breaking teda inicializácia náhodnými váhami. 
\subsubsection*{Aktivačná funkcia}
Jednou z bežne používaných aktivačných funkcií, ktorú som zvolil aj pre tento program je sigmoida. Má tvar $\varphi (x) =\frac{1}{(1+e^{-x})}$. Je vhodná kvôli jej derivovatelnosti a faktu že pre akýkoľvek vstup vracia hodnotu v rozmedzí 0 \-- 1.
\section{Dataset}
Dataset je dostupný z UCI Machine Learning Repository pod názvom Wine Quality Data Set \footnote{\url{https://archive.ics.uci.edu/ml/datasets/Wine+Quality}}. Dataset je založený na študií \footnote{\url{http://www.sciencedirect.com/science/article/pii/S0167923609001377}}, ktorá skumala možnosť predikcie preferencií ludskej chuti na základe analytických meraní chemických vlastností vína.

Atributy Datasetu:
Vstupné premenné založené na fyziochemických testoch:
\begin{enumerate}
   \item fixed acidity
   \item volatile acidity
   \item citric acid
   \item residual sugar
   \item chlorides
   \item free sulfur dioxide
   \item total sulfur dioxide
   \item density
   \item pH
   \item sulphates
   \item alcohol
\end{enumerate}
Výstupná premenná:\\
   12 - quality (score between 0 and 10)

\subsubsection*{Normalizácia}
Kvôli použitiu sigmoidy ako aktivačnej dunkcií je potrebné zvolený dataset normalizovať. Normalizácia bola vykonaná prevedeným výstupnej predikovanej hodnoty značiacej kvalitu vína z rozsahu 0 \-- 10 na rozsah 0.0 \-- 10.0. Ďaľej bol tento dataset zredukovaný o prvý riadok - hlavičku značiacu legendu jednotlivých polí.
\subsubsection*{Holdout validation} 
Dataset bol rozdelený na dva menšie datasety - trénovací a testovací. Trénovací dataset slúži k naučeniu siete, zatiaľ čo testovací je použití k overeniu správnosti naučenej siete. Pôvodný dataset bol rozdelený na trénovací a testovací v pomere 67\% a 33\% respektíve.
\subsubsection*{Predikcia}
Predikcia výslednej kvality prebieha učením siete na trénovacom datasete a následne hodnotením dát v testovacom datasete.
\subsubsection*{Rozloženie}
%TODO statistics
\section{Výsledky}
%TODO
	\subsection{Testovanie}
	%TODO	
	Počet vrstiev a neutronov
	\subsection{Štatistické výsledky}
	%TODO
\end{document}